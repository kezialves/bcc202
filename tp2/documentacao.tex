\documentclass{article}

\usepackage[brazilian]{babel}
\usepackage[letterpaper,top=2cm,bottom=2cm,left=3cm,right=3cm,marginparwidth=1.75cm]{geometry}

\usepackage{amsmath}
\usepackage{graphicx}
\usepackage{hyperref}
\usepackage{listings}
\usepackage[figure,table,lstlisting]{totalcount}
\usepackage{tcolorbox}
\usepackage{listings}
\usepackage{xcolor}

\hypersetup{
    colorlinks=true,
    linkcolor=black,
    filecolor=magenta,      
    urlcolor=cyan,
    pdftitle={Overleaf Example},
    pdfpagemode=FullScreen,
    }

\definecolor{verde}{rgb}{0.25,0.5,0.35}
\definecolor{jpurple}{rgb}{0.5,0,0.35}

\lstset{
  language=C++,
  basicstyle=\ttfamily\small,
  keywordstyle=\color{jpurple}\bfseries,
  stringstyle=\color{red},
  commentstyle=\color{verde},
  morecomment=[s][\color{blue}]{/**}{*/},
  extendedchars=true,
  showspaces=false,
  showstringspaces=false,
  numbers=left,
  numberstyle=\tiny,
  breaklines=true,
  backgroundcolor=\color{jpurple!13},
  breakautoindent=true,
  captionpos=b,
  xleftmargin=0pt,
  tabsize=4
}

\renewcommand{\lstlistingname}{Código}
\renewcommand{\lstlistlistingname}{Lista de Códigos Fonte}

\newcommand\conditionalLoF{
    \iftotalfigures
        \listoffigures
    \fi}
\newcommand\conditionalLoT{
    \iftotaltables
        \listoftables
    \fi}
\newcommand\conditionalLoL{
    \iftotallstlistings
        \lstlistoflistings
    \fi}

\newcommand{\DESCRICAO}[1]{

    \setlength{\spaceskip}{0.5em plus 1em minus 0.1em}%

    \ifdim\lastskip>0pt \hspace{0.5em plus 0.5em minus 0.1em}\fi
    \texttt{\textbf{\color{red}#1}}
}

\newcommand{\CAPA}[6]{

    \begin{titlepage}
    	
    	  \vfill
    	
    	  \begin{center}
    	    \begin{large}
    	      Universidade Federal de Ouro Preto - UFOP
    	    \end{large}
    	  \end{center}
    	
    	  \begin{center}
    	    \begin{large}
    	      Instituto de Ciências Exatas e Biológicas - ICEB
    	    \end{large}
    	  \end{center}
    	
    	  \begin{center}
    	    \begin{large}
    	      Departamento de Computação - DECOM
    	    \end{large}
    	  \end{center}
          
          \begin{center}
    	    \begin{large}
              Ciência da Computação
    	    \end{large}
    	  \end{center}
    	  
    	  \vfill
          \vfill
    	
    	  \begin{center}
    	    \begin{Huge}
    	      #1\\
    	    \end{Huge}
    	    \begin{Large}
              #2\\
    	    \end{Large}
    	  \end{center}
    	  
    	  \vfill
    	  
    	  \begin{center}
    	    \begin{large}
              #3
    	    \end{large}
    	  \end{center}

        \begin{center}
    	    \begin{large}
              #4
    	    \end{large}
    	  \end{center}

        \begin{center}
    	    \begin{large}
              #5
    	    \end{large}
    	  \end{center}
    	
    	  \begin{center}
    	    \begin{large}
    	      Professor: #6
    	    \end{large}
    	  \end{center}
    	
    	  \vfill
          \vfill
    	
    	  \begin{center}
    	    \begin{large}
    	      Ouro Preto\\
    	      \today\\
    	    \end{large}
    	  \end{center}    
    
    \clearpage

    \newpage

    \tableofcontents
    \conditionalLoF
    \conditionalLoT
    \conditionalLoL

    \end{titlepage}
}

\begin{document}

% capa

\CAPA{Ordenação de Objetos Móveis}{BCC202 - Estruturas de Dados I}{Bruno Alves Braga}{Kailainy do Patrocínio}{Kézia Alves Brito}{Pedro Silva}

% 1

\section{Introdução}

\hspace*{\parindent}Para este trabalho, é necessário entregar o código implementado na linguagem de programação C e o relatório em questão, referente ao que foi desenvolvido. O algoritmo programado consiste na ordenação de objetos móveis com base na trajetória.

% 1.1

\subsection{Especificações do problema}

\hspace*{\parindent}O objetivo deste trabalho é codificar uma solução que, através de um conjunto de trajetórias, calcule o deslocamento e a distância percorrida por \textit{n} objetos entre \textit{m} pontos, além de ordenar esses objetos com base em 3 (três) critérios: ordem decrescente da distância percorrida; crescente com base no deslocamento e ordem crescente do identificador/nome do objeto móvel.

% 1.2

\subsection{Considerações iniciais}

\hspace*{\parindent}Algumas ferramentas foram utilizadas durante a criação deste projeto:

\begin{itemize}
  \item Ambiente de desenvolvimento do código fonte: \textit{Visual Studio Code} + \textit{LiveShare}.
  \item Linguagem utilizada: C.
  \item Compilador utilizado: GCC \textit{version} 12.2.1 20230201.
  \item Ambiente de desenvolvimento da documentação: \textit{Visual Studio Code} + \LaTeX $ $ \textit{Workshop}.
\end{itemize}

% 1.3

\subsection{Ferramentas utilizadas}

\hspace*{\parindent}Algumas ferramentas foram utilizadas para testar a implementação, como:

\begin{itemize}
  \item Valgrind: ferramenta de análise dinâmica do código.
\end{itemize}

% 1.4

\subsection{Especificações da máquina}

\hspace*{\parindent}A máquina onde o desenvolvimento e os testes foram realizados possui a seguinte configuração:

\begin{itemize}
    \item[-] Processador: Intel Core i5.
    \item[-] Memória RAM: 8 GB.
    \item[-] Sistema Operacional: Manjaro Linux.
\end{itemize}

% 1.5

\subsection{Instruções de compilação e execução}

\hspace*{\parindent}Para a compilação do projeto, basta digitar:\\

\begin{tcolorbox}[title=Compilando o projeto,width=\linewidth]
    
    make

\end{tcolorbox}

Usou-se para a compilação as seguintes opções:

\begin{itemize}
    \item [-] \emph{-c}: para compilar o arquivo sem vincular os arquivos do tipo objeto.
    \item [-] \emph{-o}: para vincular os arquivos do tipo objeto.
    \item [-] \emph{-Wall}: para mostrar todos os possíveis \textit{warnings} do código.
    \item [-] \emph{-g}: para compilar com informação de depuração e ser usado pelo Valgrind.
    \item [-] \emph{-lm}: para \textit{linkar} a biblioteca \textit{math}.\\
\end{itemize}

Para a execução do programa, basta digitar:\\

\begin{tcolorbox}[title=,width=\linewidth]
    ./exe numero\_de\_objetos numero\_de\_pontos\\
    ./exe $<$ arquivo\_de\_entrada
\end{tcolorbox}

\clearpage

% 2

\section{Implementação}

\hspace*{\parindent}Em \textit{ordenacao.h} foram criados dois Tipos Abstratos de Dados - TADs, referentes aos objetos e aos pontos, necessários para a construção do programa.

\begin{lstlisting}[label={lst:cod1},language=C]
typedef struct {
    float x;
    float y;
} Ponto;

typedef struct {
    char nome[5];
    Ponto *trajetoria;
    float distancia;
    float deslocamento;
} Objeto;
\end{lstlisting}

% 2.1

\subsection{\textit{leTotalObjetos} e \textit{leTotalPontos}}

\hspace*{\parindent}Lêem o total de objetos e de pontos, retornando 0 quando for informado um valor inválido, ou seja, menor do que 1.

\begin{lstlisting}[label={lst:cod1},language=C]
int leTotalObjetos(int *totalObjetos) {

    scanf("%d", totalObjetos);

    if(totalObjetos < 1) 
        return 0;

    return 1;
}

int leTotalPontos(int *totalPontos) {
    
    scanf("%d", totalPontos);

    if(totalObjetos < 1) 
        return 0;

    return 1;
}
\end{lstlisting}

% 2.2

\subsection{\textit{alocaObjetos}}

\hspace*{\parindent}Recebe a quantidade total de objetos e de pontos e aloca, dinamicamente, um vetor de objetos, alocando também um vetor de pontos (trajetória) dentro de cada objeto, através da função \textit{alocaPontos}. Essa função retorna um vetor de TAD Objeto, que vai ser atribuído a outro vetor de mesmo tipo na \textit{main}.

\begin{lstlisting}[label={lst:cod1},language=C]
Objeto * alocaObjetos(int totalObjetos, int totalPontos) {

    Objeto *objetos = (Objeto*) malloc(totalObjetos * sizeof(Objeto));

    alocaPontos(objetos, totalObjetos, totalPontos);

    return objetos;
}

void alocaPontos(Objeto *objetos, int totalObjetos, int totalPontos) {

    for(int i = 0; i < totalObjetos; i++) {
        objetos[i].trajetoria = (Ponto*) malloc(totalPontos * sizeof(Ponto));
    }
}
\end{lstlisting}

% 2.3

\subsection{\textit{leObjetos}}

\hspace*{\parindent}Recebe um vetor de objetos alocado e lê o nome de cada objeto, assim como os pontos de sua trajétoria.

\begin{lstlisting}[label={lst:cod1},language=C]
void leObjetos(Objeto *objetos, int totalObjetos, int totalPontos) {

    for(int i = 0; i < totalObjetos; i++) {
        scanf("%s", objetos[i].nome);

        for(int j = 0; j < totalPontos; j++) {
            scanf("%f %f", &objetos[i].trajetoria[j].x, &objetos[i].trajetoria[j].y);
        }
    }
}
\end{lstlisting}

% 2.4

\subsection{\textit{teoremaPitagoras}}

\hspace*{\parindent}Para facilitar os cálculos, criamos uma função que realiza o Teorema de Pitágoras.

\begin{lstlisting}[label={lst:cod1},language=C]
float teoremaPitagoras(int x1, int y1, int x2, int y2) {

    float resultado = sqrt(pow(x1 - x2, 2) + pow(y1 - y2, 2));

    return resultado;
}
\end{lstlisting}

% 2.5

\subsection{\textit{calculaDistancia}}

\hspace*{\parindent}Calcula a distância total percorrida por um objeto dada a quantidade de pontos e as coordenadas de cada um. O cálculo é realizado a partir do somatório das distâncias entre cada par de pontos, o valor final sendo atribuído à variável \textit{distancia} da \textit{struct} de objetos.

\begin{lstlisting}[label={lst:cod1},language=C]
void calculaDistancia(Objeto *objetos, int totalObjetos, int totalPontos) {

    float distancia = 0.0;

    for(int i = 0; i < totalObjetos; i++) {
        for(int j = 0; j < totalPontos - 1; j++) {
            distancia += teoremaPitagoras(objetos[i].trajetoria[j].x,
                objetos[i].trajetoria[j].y,
                objetos[i].trajetoria[j + 1].x,
                objetos[i].trajetoria[j + 1].y);
        }

        objetos[i].distancia = distancia;
        distancia = 0.0;
    }
}
\end{lstlisting}

% 2.6

\subsection{\textit{calculaDeslocamento}}

\hspace*{\parindent}Calcula o deslocamento final de um objeto pelo Teorema de Pitágoras, realizado com as coordenadas do primeiro e último pontos por ele percorridos. Atribui o valor encontrado à variável \textit{deslocamento} dentro do TAD de \textit{Objeto}.

\begin{lstlisting}[label={lst:cod1},language=C]
void calculaDeslocamento(Objeto *objetos, int totalObjetos, int totalPontos) {

    for(int i = 0; i < totalObjetos; i++) {
        objetos[i].deslocamento = teoremaPitagoras(objetos[i].trajetoria[0].x,
            objetos[i].trajetoria[0].y,
            objetos[i].trajetoria[totalPontos - 1].x,
            objetos[i].trajetoria[totalPontos - 1].y);
    }
}
\end{lstlisting}

% 2.7

\subsection{\textit{ordenaObjetos}}

\hspace*{\parindent}Utiliza o método de ordenação \textit{Shell Sort} para ordenar um vetor de objetos em ordem decrescente de acordo com a distância total percorrida. Caso os valores de distância sejam iguais, arranja de forma crescente de acordo com os deslocamentos e, caso estes também sejam iguais, ordena alfabeticamente.

\begin{lstlisting}[label={lst:cod1},language=C]
void ordenaObjetos(Objeto *objetos, int totalObjetos) {

    int h, j;
    Objeto auxiliar;
	
    for(h = 1; h < totalObjetos;) {
        h = 3 * h + 1;
    }

    do {
        h = (h - 1) / 3;

        for(int i = h; i < totalObjetos; i++) {
            auxiliar = objetos[i];
            j = i;

            while(comparaObjetos(&objetos[j - h], &auxiliar) == 1) {
                objetos[j] = objetos[j - h];
                j = j - h;

                if(j < h) {
                    break;
                }
            }

            objetos[j] = auxiliar;
        }
    } while(h != 1);
}
\end{lstlisting}

% 2.8

\subsection{\textit{comparaObjetos}}

\hspace*{\parindent}Auxiliando a função de ordenação, a \textit{comparaObjetos} é responsável por retornar se dois valores devem (retorna 1) ou não (retorna -1) ser trocados, de acordo com os critérios pré-estabelecidos.\\
\hspace*{\parindent}Os campos que armazenam a distância e o deslocamento de cada objeto são do tipo \textit{float}, ou seja, são números de ponto flutuante. A precisão de comparação entre dois números desse tipo, em C, é de 0,01, portanto, se o módulo da diferença entre dois \textit{floats} for menor ou igual a esse valor, os dois são iguais. Caso contrário, se A - B $>$ 0,01, A $>$ B; senão, B - A $>$ 0,01 e B $>$ A.

\clearpage

\begin{lstlisting}[label={lst:cod1},language=C]
int comparaObjetos(Objeto *objeto1, Objeto *objeto2) {

    if((fabsf(objeto1->distancia - objeto2->distancia)) > 0.01) {

        if((objeto1->distancia - objeto2->distancia) > 0.01)
            return -1;

        else if((objeto2->distancia - objeto1->distancia) > 0.01)
            return 1;
    }

    else {

        if((fabsf(objeto1->deslocamento - objeto2->deslocamento)) > 0.01) {

            if((objeto1->deslocamento - objeto2->deslocamento) > 0.01)
                return 1;

            else if((objeto2->deslocamento - objeto1->deslocamento) > 0.01)
                return -1;
        }    
\end{lstlisting}

\hspace*{\parindent}Para ordenar alfabeticamente, a função utiliza a \textit{strcmp}, que retorna a soma dos valores de duas \textit{strings}. Se a soma for = 0, as \textit{strings} são iguais; senão, se a soma for $<$ 0, a segunda \textit{string} é maior; senão, a soma é $>$ 0, ou seja, a primeira \textit{string} é maior. A maior \textit{string} é a última em ordem alfabética.

\begin{lstlisting}[label={lst:cod1},language=C]
        else {

            if(strcmp(objeto1->nome, objeto2->nome) > 0)
                return 1;

            else
                return 0;
        }
    }

    return 0;
}
\end{lstlisting}

% 2.9

\subsection{\textit{imprimeVetor}}

\hspace*{\parindent}Imprime o vetor de objetos ordenado, mostrando nome, distância e deslocamento de cada objeto.

\begin{lstlisting}[label={lst:cod1},language=C]
void imprimeVetor(Objeto *objetos, int totalObjetos) {

    for(int i = 0; i < totalObjetos; i++) {
        printf("%s %.2f %.2f\n", objetos[i].nome, objetos[i].distancia, objetos[i].deslocamento);
    }
}
\end{lstlisting}

% 2.10

\subsection{\textit{desalocaObjetos}}

\hspace*{\parindent}Libera um vetor de pontos (trajetória) a partir da função \textit{desalocaPontos} e, em seguida, libera um vetor de objetos.

\begin{lstlisting}[label={lst:cod1},language=C]
void desalocaObjetos(Objeto *objetos, int totalObjetos) {

    desalocaPontos(objetos, totalObjetos);

    free(objetos);
}

void desalocaPontos(Objeto *objetos, int totalObjetos) {   

    for(int i = 0; i < totalObjetos; i++) {
        free(objetos[i].trajetoria);
    }
}
\end{lstlisting}

% 2.11

\subsection{\textit{main}}

\hspace*{\parindent}Apenas invoca e trata as respostas das funções e procedimentos definidos no arquivo \textit{ordenacao.h}.

\begin{lstlisting}[label={lst:cod1},language=C]
int main() {

    int totalObjetos, totalPontos;
    Objeto *objetos;

    if(!leTotalObjetos(&totalObjetos)) {
        printf("\nERRO: Numero de objetos invalido. O minimo eh 1.\n");
        return 0;
    }

    if(!leTotalPontos(&totalPontos)) {
        printf("\nERRO: Numero de pontos invalido. O minimo eh 1.\n");
        return 0;
    }

    objetos = alocaObjetos(totalObjetos, totalPontos);

    leObjetos(objetos, totalObjetos, totalPontos);

    calculaDistancia(objetos, totalObjetos, totalPontos);
    calculaDeslocamento(objetos, totalObjetos, totalPontos);

    if(totalObjetos != 1)
        ordenaObjetos(objetos, totalObjetos);

    imprimeVetor(objetos, totalObjetos);

    desalocaObjetos(objetos, totalObjetos);
    
    return 0;
}
\end{lstlisting}

\clearpage

% 3

\section{Impressões gerais}

\hspace*{\parindent}O algoritmo desenvolvido neste trabalho é simples e objetivo. Sendo um algoritmo de análise exploratória de dados, a ideia principal é receber um conjunto de dados e, a partir deste, resumir as características principais. Muitas vezes estas características podem ser demonstradas em métodos visuais, como um gráfico.\\\\
\hspace*{\parindent}De forma geral, a maior parte das funções implementadas são frequentemente utilizadas em outros trabalhos e atividades práticas, o que foi bem agradável.\\\\
\hspace*{\parindent}Como essa foi a primeira vez que implementamos um método de ordenação eficiente em um programa mais elaborado, no entanto, conseguimos adquirir e fixar nossos conhecimentos sobre a ordenação de um conjunto de objetos, especialmente com a utilização do \textit{Shell Sort}.

\clearpage

% 4

\section{Análise}

\hspace*{\parindent}O método de ordenação utilizado foi o \textit{Shell Sort}, que é basicamente uma extensão do \textit{Insertion Sort}. Com base nos resultados obtidos, do ponto de vista do grupo, o \textit{Shell Sort} é um dos mais eficientes métodos de ordenação, apesar de sua complexidade ser incerta. Além de eficiente, o \textit{Shell Sort} é \textit{in situ} e mais fácil de implementar em comparação a outros métodos, como o \textit{Merge Sort}, por exemplo.\\\\
\hspace*{\parindent}Além do método de ordenação, não há nenhum detalhe técnico que impacte muito no desempenho do algoritmo.
\clearpage

% 5

\section{Conclusão}

\hspace*{\parindent}O processo de implementação foi rápido e descomplicado, sendo realizado em algumas horas. Não houve muita dificuldade em entender o que deveria ser feito no trabalho, o que facilitou o início do desenvolvimento. Uma das maiores dificuldades foi escolher e implementar o melhor método de ordenação para o caso. Tentamos, inicialmente, implementar o \textit{Merge Sort}, mas logo encontramos dificuldade. Optamos, então, por implementar o \textit{Shell Sort}, que funcionou perfeitamente.\\\\
\hspace*{\parindent}Outra dificuldade encontrada durante o processo do trabalho foi a comparação entre números de ponto flutuante para ordenar o vetor com base na distância e deslocamento, uma vez que, na linguagem C, há uma margem de erro na precisão das casas decimais.

\end{document}